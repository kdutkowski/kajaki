% This is based on the LLNCS.DEM the demonstration file of
% the LaTeX macro package from Springer-Verlag
% for Lecture Notes in Computer Science,
% version 2.4 for LaTeX2e as of 16. April 2010
%
% See http://www.springer.com/computer/lncs/lncs+authors?SGWID=0-40209-0-0-0
% for the full guidelines.
%

\documentclass{llncs}

\usepackage[utf8]{inputenc}
\usepackage[T1]{fontenc}
\usepackage[polish]{babel}
\usepackage[colorinlistoftodos]{todonotes}
\usepackage{hyperref}
\usepackage{subfig}
\usepackage[]{algorithm2e}

\usepackage{float}

\usepackage{listings}
\usepackage{courier}
 \lstset{
         basicstyle=\footnotesize\ttfamily,
         xleftmargin=.4\textwidth, xrightmargin=.2\textwidth
 }

\begin{document}

\title{Rozwiązanie problemu kajaków}
%
\titlerunning{Kajaki}  % abbreviated title (for running head)
%                                     also used for the TOC unless
%                                     \toctitle is used
%
\author{Tomasz Janiszewski\inst{1} \and Jakub Dutkowski\inst{2}}
%
\authorrunning{Tomasz Janiszewski, Jakub Dutkowski} % abbreviated author list (for running head)
%
%%%% list of authors for the TOC (use if author list has to be modified)
\tocauthor{Tomasz Janiszewski and Jakub Dutkowski}

\institute{Politechnika Warszawska\\
Wydział Matematyki i Nauk Informacyjnych
\email{janiszewskit@student.mini.pw.edu.pl}~~
\email{dutkowskij@student.mini.pw.edu.pl}}

\maketitle              % typeset the title of the contribution

\keywords{Kajaki}
%
\section{Treść zadania}
Są dwa kajaki, $n$ osób i zbiór par osób które chcą się
przepłynać kajakiem razem (jedna osoba może kolejno z wieloma osobami się
przepłynać). Zastosować algorytm znajdujący skojarzenie doskonałe do
wyznaczenia rozkładu jazdy kajaków, tak aby zminimalizować ilość kursów
pojedynczym kajakiem, spełniając życzenia wszystkich osób.

\subsection{Opis wejścia}
W pierwszej lini wejścia znajdują się dwie nieujemne liczby ($n, m$) oznaczające odpowiednio
ilość osób oraz liczbę par osób chcących popłynąć razem.
W kolejnych $m$ liniach znajdują sie pary liczb oznaczające, że osoba o numerze $m_{i,1}$ chce się
przepłynąć z $m_{i,2}$

\subsection{Opis wyjścia}
Na wyjściu znajduje się jedna liczba będąca maksimum z liczby kursów każdego z kajaków.


\begin{example} \label{przyklad} ~\\
\begin{lstlisting}[title=Wejście]
6 7
1 2
1 6
2 3
2 6
3 5
4 5
5 6
\end{lstlisting}
\begin{lstlisting}[title=Wyjście]
4
\end{lstlisting}

\begin{lstlisting}[title=Przykładowy rozkład plywania]
1 6	2 3
5 4	1 2
5 6	 -
2 6	3 5
\end{lstlisting}

\end{example}

\section{Algorytm}
Powyższy problem można rozwiązać następującym algorytmem:

\subsection{Opis}
Budujemy graf opisujący dane wejściowe. Będziemy w nim szukali 
maksymalnego skojarzenia, które zdefiniuje schemat pływania.
W każdym kroku po znalezieniu skojarzenia będziemy usuwać 
użyte wierzchołki i powtarzać krok aż do uzyskania grafu bez krawędzi.

\subsubsection{Budowa grafu}
Graf uzyskamy w następujący sposób. Budujemy
graf $G''$ w którym każdej osobie odpowiada pojedynczy 
wierzchołek. Wierzchołki są połączone krawędzią wtedy
i tylko wtedy gdy odpowiadające im dwie osoby chcą płynąć razem.
Następnie tworzymy graf krawędziowy $G'$ z grafu $G''$.

Ostatnim krokiem jest wyznaczenie grafu $G$
o zbiorze wierzchołków równym $V(G')$, ale krawędzią
połączone będą tylko te wierzchołki, między którymi w grafie $G'$
istnieje ścieżka długości 2 i nie istnieje ścieżka długości 1.
\\\\
W tak otrzymanym grafie $G$ szukamy największego 
skojarzenia. Znalezione krawędzie definują
pary, które mogą razem płynąć. Następnie usuwamy użyte
wierzchołki z grafu i powtarzamy krok.
Algorytm kończymy gdy graf $G$ nie ma już żadnej krawędzi.
Pozostałe wierzchołki dopisujemy do listy płynących
osób.
\begin{example}~\\
Posłużymy się danymi z przykładu \ref{przyklad}.
Budujemy graf określający preferencje pływania.
\begin{figure}[H]
  \caption{Graf $G''$}
  \centering
	\includegraphics[scale=0.5]{graph/example_2_step_1}
\end{figure}
Konstruujemy graf krawędziowy $G''$
\begin{figure}[H]
  \caption{Graf $G'$}
  \centering
	\includegraphics[scale=0.5]{graph/example_2_step_2}
\end{figure}
Wyznaczamy graf $G$
\begin{figure}[H]
  \caption{Graf $G$}
  \centering
	\includegraphics[scale=0.5]{graph/example_2_step_3}
\end{figure}

W grafie $G$ szukamy maksymalnego skojarzenia (oznaczone kolorem czerwonym).
\begin{lstlisting}[title=Maksymalne skojarzenie]
1 6	3 5
2 3	1 6
5 4	2 6
\end{lstlisting}
Po usunięciu wybranych wierzchołków pozostaje nam tylko jeden 
wierzchołek $5|6$.
\begin{lstlisting}[title=Możliwy rozkład pływania]
1 6	3 5
2 3	1 6
5 4	2 6
5 6	 -
\end{lstlisting}

\end{example}

\subsection{Pesudokod}
\begin{algorithm}[H]
 \caption{Algorytm wyznaczania rokładu pływania kajaków}
 \KwData{$G$ -- graf przedstawiający preferencje pływania\;
 $V$ -- funkcja zwracająca listę wierzchołków podanego grafu\;
 $E$ -- funkcja zwracająca listę krawędzi podanego grafu\;
 }
 \KwResult{Rozkład pływania kajaków zapisany w listach $k_1, k_2$}
 $k_1$ = pusta lista krawędzi\;
 $k_2$ = pusta lista krawędzi\;
 \While{$E(G) \neq \emptyset$}{
  $s$ = skojarzenie maksymalne grafu $G$\;
  $G = G - s$\;
  \While{$s \neq \emptyset$}{
    $pair = s.pop()$\;
  	$k_1.push(pair.first())$\;
  	$k_2.push(pair.second())$\;
  }
 }
 \While{$V(G) \neq \emptyset$}{
 	$v = V(G).pop()$\;
 	\tcc{kolizja - istnienie tego samego wierzchołka na obu listach na tej samej pozycji.}
	\eIf{$v$ nie koliduje z $k_2$ i $len(k_2) > len(k_1)$}{
   		$k_1.push(v)$\;
     }{
		$k_2.push(v)$\;
  	 }  	 	
 }  	
\end{algorithm}

\begin{proof}
Załóżmy, że powyższy algorym nie generuje optymalnego rozwiązania, to znaczy istnieje
takie rozwiązanie w którym długość listy $k_1$ lub $k_2$ jest mniejsza niż znaleziona wartość.
Gdyby tak było to oznaczałoby to że jednym z przejść pętli istnieje
skojarzenie zawierające więcej krawędzi niż skojarzenie $s$. 
Oznaczało by to, że $s$ nie jest maksymalne/poprawne i w rezultacie jedna z list $k_1$ lub $k_2$ byłaby wypełniana pustymi
symbolami (oznaczającymi czekanie w danej rundzie). 
\\\\
Ponadto konstrukcja grafu $G$ zapewnia, że żadna z par wierzchołków należących do skojarzenia znalezionego w tym grafie nie będzie zawierała żadnego wierzchołka więcej niż raz - tj. nie będzie powodowała kolizji. Wynika to z tego, że w grafie $G$ krawędzie łączą tylko wierzchołki odpowiadające krawędziom grafu $G''$ nie posiadającym wspólnego wierzchołka.
\end{proof}

\subsection{Oszacowanie złożoności}
Budowanie grafu $G$ wymaga zbudowania grafu krawędziowego i znalezienie w nim ścieżek długości 2. 


Zbudowanie grafu krawędziowego $G''$
wymaga przejścia przez wszystkie krawędzie grafu i sprawdzenia, czy ma punkty wspólne z każdą inną krawędzią co można 
szacować jako $O(E^2)$. 


Aby wyznaczyć graf $G'$ wystarczy podnieść macierz sąsiedztwa do kwadratu i porównać z oryginalną macierzą sąsiedztwa. Złożoność szacujemy jako $O(V^3)$.

 
Wyszukiwanie maksymalnego skojarzenia grafu możemy oszacować przez $O(|V|^4)$\cite{wiki-blossom}.
Liczbę takich przeszukiwań możemy z góry ograniczyć przez liczbę krawędzi czyli $|E|$. 
Pozostałe operacje takie jak sprawdzenie czy krawędź nie koliduje z rozwiązaniem już znalezionym
oraz dodawanie, zdejmowanie i sprawdzanie elementu na liście wykonujemy w czasie stałym $O(1)$.
Daje to nam sumaryczną złożoność $O(|E^2|+|V^3|+|E||V|^4)$ co można oszacować z góry jako 
\begin{equation}
O(|V|^6)
\end{equation}
%
% ---- Bibliography ----
%
\begin{thebibliography}{4}
%
\bibitem {cormen}
Cormen, Thomas H. and Stein, Clifford and Rivest, Ronald L. and Leiserson, Charles E.:
\textsl{Introduction to Algorithms}
McGraw-Hill Higher Education, 2001
\bibitem{mucha-sankowski}
Mucha, Marcin and Sankowski, Piotr:
\textsl{\href{http://www.mimuw.edu.pl/~mucha/pub/mucha_sankowski_focs04.pdf}{Maximum Matchings via Gaussian Elimination}}
\bibitem{wiki-blossom}
Wikipedia:
\textsl{\href{http://en.wikipedia.org/wiki/Blossom_algorithm}{Edmonds's matching algorithm}}
\end{thebibliography}

\end{document}
