% This is based on the LLNCS.DEM the demonstration file of
% the LaTeX macro package from Springer-Verlag
% for Lecture Notes in Computer Science,
% version 2.4 for LaTeX2e as of 16. April 2010
%
% See http://www.springer.com/computer/lncs/lncs+authors?SGWID=0-40209-0-0-0
% for the full guidelines.
%

\documentclass{llncs}

\usepackage[utf8]{inputenc}
\usepackage[T1]{fontenc}
\usepackage[polish]{babel}
\usepackage[colorinlistoftodos]{todonotes}
\usepackage{hyperref}
\usepackage{subfig}

\begin{document}

\title{Rozwiązanie problemu kajaków}
%
\titlerunning{Kajaki}  % abbreviated title (for running head)
%                                     also used for the TOC unless
%                                     \toctitle is used
%
\author{Tomasz Janiszewski\inst{1} \and Jakub Dutkowski\inst{2}}
%
\authorrunning{Tomasz Janiszewski, Jakub Dutkowski} % abbreviated author list (for running head)
%
%%%% list of authors for the TOC (use if author list has to be modified)
\tocauthor{Tomasz Janiszewski and Jakub Dutkowski}

\institute{Politechnika Warszawska\\
Wydział Matematyki i Nauk Informacyjnych
\email{janiszewskit@student.mini.pw.edu.pl}~~
\email{dutkowskij@student.mini.pw.edu.pl}}

\maketitle              % typeset the title of the contribution

\keywords{Kajaki}
%
\section{Treść zadania}
Są dwa kajaki i n osób i zbiór par osób które chcą się
przepłynać kajakiem razem (jedna osoba może kolejno z wieloma osobami się
przepłynać). Zastosować algorytm znajdujący skojarzenie doskonałe do
wyznaczenia rozkładu jazdy kajaków, tak aby zminimalizować ilość kursów
pojedynczym kajakiem, spełniając życzenia wszystkich osób.

%
% ---- Bibliography ----
%
\begin{thebibliography}{4}
%
\bibitem {cormen}
Cormen, Thomas H. and Stein, Clifford and Rivest, Ronald L. and Leiserson, Charles E.:
\textsl{Introduction to Algorithms}
McGraw-Hill Higher Education, 2001
\end{thebibliography}

\end{document}
